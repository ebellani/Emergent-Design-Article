\section{Conclusion}

The technical conditions for an education that is modeled in the framework detailed in the
course of this article are already available, and some initial implementations for
topics such as language acquisition are already available on the market.
\cite{education:anne_eisenberg_learning_native_speaker} Those could be
adapted as models to be used in a very large scale for any number of topics.

Some of the benefits of learning in this environment were demonstrated in the
introduction, but to make some more emphasis, we can see some ``key ingredients
for creating rich design environments'', that
\cite{education:resnick&ocko_learning_through_design} describe for their
LEGO/Logo environment for children learning, and that , correlates with the
system described in this article:

\begin{itemize}
    \item Put people in control.
    \item Offer Multiple paths to learning
    \item Encourage a sense of community
\end{itemize}

But perhaps even more steep will be the social conditions for developing this
system, seeing that it is mainly a social machine. Even more if we consider what
\cite{education:negroponte_speech_ted} has to say about it:

\begin{quote} 
    If you look at governments around the world, ministries of education tend to be
    the most conservative, and also the ones that have huge payrolls, everybody
    thinks they know about education, a lot of culture is built into it as well,
    it's really hard.
\end{quote} 

The solution for those situations are completely out of the scope of this
article, but personal effort and conscious struggle for a better
education is a sure step on the right direction. Plus, this system provokes the
emergence of self-reinforcement, in the sense that the more a person is is
exposed and trusted to control his/her own learning, the more engaged and
enthusiastic he/she will be about the process. To illustrate this pattern,
consider \cite{education:wilensky_abstract_meditations_concrete} though: 
\clearpage%so it don't break the quote
\begin{quote} 
    What kinds of relationships between people would be fostered by a society
    which stipulated that people be introduced to each other formally and
    thereafter relate only in prescribed, rule-driven ways? If you shudder at
    this prospect, consider the analogy between this scenario and the
    instructionist paradigm for learning (see
    \cite{education:papert&harel_software_design_learning_enviroment}). It is
    through people's own idiosyncratically personal ways of connecting to other
    people that meaningful relationships are established. In a similar way, when
    learners are in an environment in which they construct their own
    relationships with the objects of knowledge, these relationships can become
    deeply meaningful and profound.   
\end{quote}


This article ends with the wise
words of \cite{education:resnick&ocko_learning_through_design} about it:

\begin{quote} 
    Implementing these strategies is not easy. There are many unanswered
    questions--such as how to help students ``break away'' from their initial
    ``regions of comfort.'' And coordinating open-ended design activities is a
    challenge for any teacher. Indeed, organizing an ``Inventor's Workshop'' is
    far more difficult than delivering a lecture on mechanical advantage, or
    developing a step-by-step hands-on lesson. As
    \cite{education:dewey_experience_education} noted more than a half-century
    ago (in words that still ring true today): ``The road of the new education
    is not an easier one to follow than the old road but a more strenuous and
    difficult one.'' But it is a road well worth taking. 
\end{quote}

