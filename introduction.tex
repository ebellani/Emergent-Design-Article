\section{Introduction}


Two fundamental concepts form the basis of the educational system this article is trying to
communicate. One is constructionism, a theory developed by Seymour Papert, and
it is the epistemological approach that runs throughout the system, and affects
how the learning (and consequently the teaching) process is viewed. The second
one is emergent design, and that is the organizational and structural view of
the system, and affects how the system evolves and relates to its members and
environment. Both concepts are intertwined, as will be shown below. 

\subsection{Constructionism}
Constructionism is a epistemological theory that was developed by Seymour
Papert, drawing from his experiences as a mathematician, his time studying with
Jean Piaget (the developer of the constructivist theory) in France and from his
background as an artificial intelligence and media researcher.

\cite{education:cavallo_building_knowledge} makes a clear distinction between Constructionism and
Constructivism:

\begin{quote}
  Constructionism builds upon principles in constructivism. While constructivism
  holds that the learner constructs new knowledge based on the existing knowledge
  he or she has, constructionism builds on this idea by maintaining that this
  process happens particularly well when the learner is in the process of
  constructing something.                   
  
\end{quote}

Constructionism is also a true pedagogy of autonomy, in a very Freirean way:
\begin{quote}
...teaching is not transferring knowledge, but creating opportunities for their own
production or its construction.
\cite{education:paulo_freire__pedagogia_da_autonomia}
\end{quote}

We can observe the deep connection of constructionism with computers and
technology. But that is far from a technocentrist, where technology is used just
for the sake of using technology. That connection comes from the almost endless
capacity of the computer to create simulations and concept of microworlds:

\begin{quotation}
Learners in a physics microworld are able to invent their own personal sets of
assumptions about the microworld and its laws and are able to make them come
true.  They can shape the reality in which they will work for the day, they can
modify it and build alternatives.  This is an effective way to learn,
paralleling the way in which each of us once did some of our most effective
learning. Piaget has demonstrated that children learn fundamental mathematical
ideas by first building their own, very much different (for example,
preconservationist) mathematics. And children learn language by first learning
their own ("baby-talk") dialects. So, when we think of microworlds as incubators
for powerful ideas, we are trying to draw upon this effective strategy: We allow
learners to learn the "official" physics by allowing them the freedom to invent
many that will work in as many invented worlds.
\cite{education:papert_mindstorms}
\end{quotation}

That kind of characteristic is crucial to deeper changes in education and what
is meant by knowing. 

\begin{quotation}
The need to distinguish between a first impact on education and a deeper meaning
is as real in the case of computation as in the case of feminism. For example,
one is looking at a clear case of first impact when "computer literacy" is
conceptualized as adding new content material to a traditional curriculum.
Computer-aided instruction may seem to refer to method rather than content, but
what counts as a change in method depends on what one sees as the essential
features of the existing methods. From my perspective, CAI amplifies the rote
and authoritarian character that many critics see as manifestations of what is
most characteristic of--and most wrong with--traditional school. Computer
literacy and CAI, or indeed the use of word-processors, could conceivably set up
waves that will change school, but in themselves they constitute very local
innovations--fairly described as placing computers in a possibly improved but
essentially unchanged school. The presence of computers begins to go beyond
first impact when it alters the nature of the learning process; for example, if
it shifts the balance between transfer of knowledge to students (whether via
book, teacher, or tutorial program is essentially irrelevant) and the production
of knowledge by students. It will have really gone beyond it if computers play a
part in mediating a change in the criteria that govern what \emph{kinds of knowledge
are valued in education.} 
\cite{education:papert__situating_constructionism}
\end{quotation}


\subsection{Emergent Design}

Emergent design is a strategy for building systems that are too complex to be
tackled in a top down, plan first manner, as
\cite{education:cavallo__technological_fluency} puts is:
\begin{quotation}
Popular views about design, about reform, about planning, about control
typically lag behind progress. New organizations are pioneering new means of
control and change.  Emergent design is the recognition that certain systems are
too complex, dynamic, interconnected, and chaotic to attempt to manage them by
top-down, pre-planned, rigid means of control. Large educational systems are
one-such system. The human brain is another.  That this project is
simultaneously involved with both systems is all the more reason to take an
emergent approach.
\end{quotation}

So the point of the strategy is to build systems that provide a base for systems
to be built on it, by the users of the system itself. A global participatory architecture
that connects several systems that grew on top of it. That view has several
advantages as xxx objectives for workshops in his emergent design system in
Thailand shows:

\begin{quote}
  The workshops were intended to:
  \begin{itemize}

 \item provide powerful personal
experiences of a different approach to learning,

 \item break pessimistic mindsets about people’s ability to learn,

 \item surface, reflect upon and discuss participants’ own prior explicit and
 implicit assumptions about learning to the surface, and compare them to the new
 experience,

 \item encourage participants to think about the learning process itself,

 \item engage in thinking about the design and practice of learning environments
 in the local context,

 \item identify local people whose thinking and acting appear promising so that
 they can take on greater roles for change,

 \item debug our own thinking about the mechanisms of learning and our own
 pattern of practice in designing learning environments.  

 \end{itemize}
\end{quote}

\subsection{Why}

But why propose of a new system for higher education, one that puts the learner
in the driver seat, that can growth organically with a community and is deeply
connected to what is important for an individual and his community? Because I see that the act of
descholling society, in the sense of individuals empowering themselves, is
critical for the current and future society that wishes to remain democratic.

\cite{education:cavallo__models_of_growth} quotes John Dewey on exactly this fact:
\begin{quote}
For Dewey a just society could only be built not based upon the dictates of
clergy, royalty, or an elite, but depended upon the informed collective
decisions of all, where every voice should be heard.
\end{quote}

But what descholling means, and to who falls the responsibility for deschooling
an individual? \cite{education:ivan_illich__deschooling_society} provides a
clear answer to both questions.

\begin{quotation}
  Only liberating oneself from school will dispel such illusions. The discovery
  that most learning requires no teaching can be neither manipulated nor planned.
  Each of us is personally responsible for his or her own deschooling, and only we
  have the power to do it. No one can be excused if he fails to liberate himself
  from schooling. People could not free themselves from the Crown until at least
  some of them had freed themselves from the established Church. They cannot free
  themselves from progressive consumption until they free themselves from
  obligatory school.
\end{quotation}

So if we analyze that the current and next generations of children, poor and rich alike, are
growing along with computers and their powerful integration capabilities, and
the aim of projects like the OLPC is: 

\begin{quote}
    to adequately educate all the children of the
    emerging world. Simply doing more of the same is no longer enough, if it
    ever was. If their citizens are to benefit, as they should from the spread
    of the technology-based, global information economy, these nations must
    rethink the old top-down classroom paradigm. 
    \cite{education:olpc_educational_proposition}
\end{quote}

What can we say about the next educational step of for those children, higher
education, in this context?
Certainly a person who grew used to those conditions in their development 
would find the current state of affairs in university appalling. 

\cite{futurism:kurzweil_singularity_is_near}, in his last best selling futurist
book tries to make a point about the importance of the exponential pace of
technological development, and how it will affect the individual and the
society as a whole on all conceivable aspects on a very close future.  A
society that does not prepare it's citizens for such future is doomed to become
mere spectator and follower of those who do. A small account of his view of the
present educational situation perhaps illustrates the point better:

\begin{quotation}
    Most education in the world today, including in the wealthier communities,
    is not much changed from the model offered by the monastic schools of
    fourteenth-century Europe. Schools remain highly centralized institutions
    built upon the scarce resources of buildings and teachers. The quality of
    education also varies enormously, depending on the wealth of the local
    community (the American tradition of funding education from property taxes
    clearly exacerbates this inequality), thus contributing to the have/have not
    divide.
\end{quotation}

The impact of delaying, however, is not only social, but profoundly individual as well, as
\cite{education:papert_gaston_vision_for_education} puts it, ``So the choice is
not whether we will consider deep changes in school but how many children will
be lost before we recognize that we have to do so'', or in other words on the
same work:
\begin{quotation}
    As the slow evolution of school lags further and further behind the rapid
    evolution of society, increasing numbers of students all over the world see
    school as irrelevant to life. Many drop out. Many more drop out mentally,
    emerging from school with poor skills and negative visions of themselves and the
    society they are entering.
\end{quotation}
