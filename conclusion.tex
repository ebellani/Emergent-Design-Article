%    Thus, conclusions should be short and sweet. Do not restate all of your findings. One
%statement in the abstract, one in the introduction and once more in the body of the text
%should be enough! You can include a short paragraph or two acknowledging limitations,
%suggesting implications beyond those in the paper. Keep it short though — don’t write your
%grant application here outlining all of your plans for future research. And don’t speculate;
%the reader wants to know your facts not your opinions.

%Conclusion
%
%To summarize, I am presenting for computer science an argument that has come up before in many other areas of science and mathematics education. Our official curricula, I think, run the same risk as the official physics curriculum that caused Albert Einstein to drop out of school. Such rigidity would be particularly regrettable in a subject like computer programming, which lends itself so readily to a flexible, experimental approach. 
%
%

\section{Conclusion}

Summarizing, I am presenting an argument for an emergent design approach for
education systems, one that is based on an approach for emergent design detailed
by \cite{education:cavallo__technological_fluency} and that uses as its
pedagogical and epistemological base constructionism, as situated in
\cite{education:papert__situating_constructionism}.

This article ends with the words of
\cite{education:resnick&ocko_learning_through_design} that I find very relevant
for systems like the one I proposed so far:

\begin{quote} 
    Implementing these strategies is not easy. There are many unanswered
    questions--such as how to help students ``break away'' from their initial
    ``regions of comfort.'' And coordinating open-ended design activities is a
    challenge for any teacher. Indeed, organizing an ``Inventor's Workshop'' is
    far more difficult than delivering a lecture on mechanical advantage, or
    developing a step-by-step hands-on lesson. As
    \cite{education:dewey_experience_education} noted more than a half-century
    ago (in words that still ring true today): ``The road of the new education
    is not an easier one to follow than the old road but a more strenuous and
    difficult one.'' But it is a road well worth taking. 
\end{quote}

