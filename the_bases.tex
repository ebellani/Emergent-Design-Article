\section{The bases of the system}
Two fundamental concepts form the basis of the educational system that I am
trying to communicate in this article. One is constructionism, a theory
developed by Seymour Papert, and it is the epistemological approach that runs
throughout the system, and affects how the learning (and consequently the
teaching) process is viewed.  The second one is emergent design, and that is the
organizational and structural view of the system, and affects how the system
evolves and relates to its members and environment. Both concepts are
intertwined, as will be shown below. 

\subsection{Constructionism}
Constructionism is an epistemological theory that was developed by Seymour
Papert, drawing from his experiences as a mathematician, his time studying with
Jean Piaget (the developer of the constructivist theory) in France and from his
background as an artificial intelligence and media researcher.

\cite{education:cavallo_building_knowledge} makes a clear distinction between Constructionism and
Constructivism:

\begin{quote}
    Constructionism builds upon principles in constructivism. While constructivism
    holds that the learner constructs new knowledge based on the existing knowledge
    he or she has, constructionism builds on this idea by maintaining that this
    process happens particularly well when the learner is in the process of
    constructing something.                   
\end{quote}

Constructionism is also a true pedagogy of autonomy, in a very Freirean way:
\begin{quote}
    ...teaching is not transferring knowledge, but creating opportunities for their own
    production or its construction.
\cite{education:paulo_freire__pedagogia_da_autonomia}
\end{quote}

We can observe the deep connection of constructionism with computers and
technology. But that is far from a technocentrist, where technology is used just
for the sake of using technology. That connection comes from the almost endless
capacity of the computer to create simulations and concept of microworlds:

\begin{quotation}
    Learners in a physics microworld are able to invent their own personal sets of
    assumptions about the microworld and its laws and are able to make them come
    true.  They can shape the reality in which they will work for the day, they can
    modify it and build alternatives.  This is an effective way to learn,
    paralleling the way in which each of us once did some of our most effective
    learning. Piaget has demonstrated that children learn fundamental mathematical
    ideas by first building their own, very much different (for example,
    preconservationist) mathematics. And children learn language by first learning
    their own ("baby-talk") dialects. So, when we think of microworlds as incubators
    for powerful ideas, we are trying to draw upon this effective strategy: We allow
    learners to learn the "official" physics by allowing them the freedom to invent
    many that will work in as many invented worlds.
\cite{education:papert_mindstorms}
\end{quotation}

That kind of characteristic is crucial to deeper changes in education and what
is meant by knowing. 

\begin{quotation}
    The need to distinguish between a first impact on education and a deeper meaning
    is as real in the case of computation as in the case of feminism. For example,
    one is looking at a clear case of first impact when "computer literacy" is
    conceptualized as adding new content material to a traditional curriculum.
    Computer-aided instruction may seem to refer to method rather than content, but
    what counts as a change in method depends on what one sees as the essential
    features of the existing methods. From my perspective, CAI amplifies the rote
    and authoritarian character that many critics see as manifestations of what is
    most characteristic of--and most wrong with--traditional school. Computer
    literacy and CAI, or indeed the use of word-processors, could conceivably set up
    waves that will change school, but in themselves they constitute very local
    innovations--fairly described as placing computers in a possibly improved but
    essentially unchanged school. The presence of computers begins to go beyond
    first impact when it alters the nature of the learning process; for example, if
    it shifts the balance between transfer of knowledge to students (whether via
    book, teacher, or tutorial program is essentially irrelevant) and the production
    of knowledge by students. It will have really gone beyond it if computers play a
    part in mediating a change in the criteria that govern what \emph{kinds of knowledge
    are valued in education.} 
\cite{education:papert__situating_constructionism}
\end{quotation}


\subsection{Emergent Design}

Emergent design is a strategy for building systems that are too complex to be
tackled in a top down, plan first manner, as
\cite{education:cavallo__technological_fluency} puts is:
\begin{quotation}
Popular views about design, about reform, about planning, about control
typically lag behind progress. New organizations are pioneering new means of
control and change.  Emergent design is the recognition that certain systems are
too complex, dynamic, interconnected, and chaotic to attempt to manage them by
top-down, pre-planned, rigid means of control. Large educational systems are
one-such system. The human brain is another.  That this project is
simultaneously involved with both systems is all the more reason to take an
emergent approach.
\end{quotation}

So the point of the strategy is to build systems that provide a base for systems
to be built on it, by the users of the system itself. A global participatory architecture
that connects several systems that grew on top of it. That view has several
advantages as \cite{education:cavallo__technological_fluency} objectives for workshops in his emergent design system in
Thailand shows:

\begin{quotation}
  The workshops were intended to:
    \begin{itemize}

        \item provide powerful personal
        xperiences of a different approach to learning,

        \item break pessimistic mindsets about people’s ability to learn,

        \item surface, reflect upon and discuss participants’ own prior explicit and
        implicit assumptions about learning to the surface, and compare them to the new
        experience,

        \item encourage participants to think about the learning process itself,

        \item engage in thinking about the design and practice of learning environments
        in the local context,

        \item identify local people whose thinking and acting appear promising so that
        they can take on greater roles for change,

        \item debug our own thinking about the mechanisms of learning and our own
        pattern of practice in designing learning environments.  

    \end{itemize}
\end{quotation}
