\section{Introduction}

The fundamental problem that I tackle in this article is putted by \cite{sagan_1996__candle_in_dark}  
very eloquently, so I'll quote him in full:

\begin{quotation}
    We've arranged a global civilization in which most crucial elements -
    transportation, communications, and all other industries; agriculture,
    medicine, education, entertainment, protecting the environment; and even
    the key democratic institution of voting - profoundly depend on science and
    technology. We have also arranged things so that almost no one understands
    science and technology. This is a prescription for disaster. We might get
    away with it for a while, but sooner or later this combustible mixture of
    ignorance and power is going to blow up in our faces.
\end{quotation}

Or consider the synthesis that \cite{education:cavallo__models_of_growth} makes
about the prophetic vision of John Dewey on justice and society:

\begin{quote}
    For Dewey a just society could only be built not based upon the dictates of
    clergy, royalty, or an elite, but depended upon the informed collective
    decisions of all, where every voice should be heard.
\end{quote}

So, clearly our current ways of learning and teaching are broken, but why? I
stress that this is because of the centralized nature of the responsible
institutions. It is the nature of centralized process to not being able to cope
with change. As \cite{futurism:kurzweil_singularity_is_near} puts it:

\begin{quotation}
    Most education in the world today, including in the wealthier communities,
    is not much changed from the model offered by the monastic schools of
    fourteenth-century Europe. Schools remain highly centralized institutions
    built upon the scarce resources of buildings and teachers. The quality of
    education also varies enormously, depending on the wealth of the local
    community (the American tradition of funding education from property taxes
    clearly exacerbates this inequality), thus contributing to the have/have not
    divide.
\end{quotation}

Unfortunately compounding this problem is the fact that the bureaucracies
responsible for the education in most, if not all, societies in the world tend
to be the most resistant to change, where exists a heavy payroll and people
think they know everything about education. \cite{education:negroponte_speech_ted}

I'll call the process of decentralizing of the educational systems deschooling,
paying homage to \cite{education:ivan_illich__deschooling_society}, because I
think he was one of the first who saw it clearly needed in modern society.  But
what descholling means, and to whom falls the responsibility for deschooling an
individual? The same work provides a clear answer to both questions.

\begin{quotation}
    Only liberating oneself from school will dispel such illusions. The discovery
    that most learning requires no teaching can be neither manipulated nor planned.
    Each of us is personally responsible for his or her own deschooling, and only we
    have the power to do it. No one can be excused if he fails to liberate himself
    from schooling. People could not free themselves from the Crown until at least
    some of them had freed themselves from the established Church. They cannot free
    themselves from progressive consumption until they free themselves from
    obligatory school.
\end{quotation}

The impact of keeping the course is huge, both socially and individually, as
\cite{education:papert_gaston_vision_for_education} thinks: 

\begin{quotation} 
    The choice is not whether we will consider deep changes in school but how many
    children will be lost before we recognize that we have to do so.
\\
...
\\
    As the slow evolution of school lags further and further behind the rapid
    evolution of society, increasing numbers of students all over the world see
    school as irrelevant to life. Many drop out. Many more drop out mentally,
    emerging from school with poor skills and negative visions of themselves and the
    society they are entering.
\end{quotation}

In the rest of the article I'll try to present a system that uses current
technologies and networks to enable individuals to take charge of their own
education, along with a path for them to prove their acquired knowledge to society
in general. It is the objective of the article to make clear that, by providing
the conditions to a system with the characteristics outlined in the rest of the
article the problems described in this introduction would be considerably
mitigated.

